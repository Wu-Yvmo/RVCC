\documentclass[a4paper, oneside, twocolumn]{ctexbook}
\usepackage{ctex}

\begin{document}

\author{吴雨墨}
\title{RV64GC指令选择}
\maketitle

\tableofcontents

\chapter{RV64GC指令}
本章节中阐述RV64GC的指令体系。这些指令中,一部分是真实存在的指令(描述于
第一小节),而另一部分是所谓的“伪指令”(描述于第二小节)。此外,RV64GC体系中,一些
指令还存在不同宽度的版本。比如MUL、DIV、REM、ADD和SUB指令就存在W后缀。W后缀表示
执行32位运算,在没有后缀时指令会执行64位运算。而Load和Store类指令则提供了更为精细
的位宽控制,分别有LB、LH、LW、LD、SB、SH、SW和SD。在所有后缀中,B表示1字节,H表示2字节,W表示4
字节。不同于算数类指令,存取类指令没有默认的位宽,也必须有后缀。

\section{存取类指令}
RV64GC作为典型的RISC指令集,也遵循RISC的基本规则,只有专门的存取指令能够读写内存,其他
非存取指令只能操作寄存器或立即数。RV64GC的存取类指令分为Load和Store两类,
其中Load类指令从内存中加载数据,Store类指令将寄存器中的数据写入
到内存中。
RV64GC的Store指令为:SD、SW、SH和SB,它们的功能是将寄存器中的
全部数据、低4字节、低2字节和低1字节写入到内存中。
RV64GC的Load指令为:LD、LW、LH和LB,它们的功能是将内存中的
全部数据、低4字节、低2字节和低1字节写入到寄存器中。

在RV64GC-ISA中,Store类指令的GNU汇编格式是:
\begin{center}
\verb+s{d|w|h|b} rs, offset(rb)+
\end{center}

其中,\verb|rs|是源寄存器,负责提供数据,\verb|offset|是偏移量,
而\verb|rb|是基址寄存器。该指令会将\verb|rs|中的数据写入到地址
为\verb#rb+offset#的内存中。

在RV64GC-ISA中,Load类指令的GNU汇编格式是:
\begin{center}
\verb+l{d|w|h|b} rd, offset(rb)+
\end{center}

其中,\verb|rd|是目的寄存器,负责接收数据,\verb|offset|是偏移量,
而\verb|rb|是基址寄存器。该指令会将地址
为\verb#rb+offset#的内存中的数据读入到\verb|rd|中。

\section{运算类指令}
常用的运算类指令有MUL、DIV、ADD和SUB。
这些指令都是有符号的。运算类指令默认是64位运算。
32位版本的运算类指令是:MULW、DIVW、ADDW和SUBW。
在RV64GC中,运算类指令的GNU汇编的格式是:

\begin{center}
\verb+{add|sub|mul|div}{w} rd, rs1, rs2+
\end{center}

其中\verb|rd|是目的寄存器,负责接收数据,\verb|rs1|是左操作数,
\verb|rs2|是右操作数。

\section{逻辑类指令}

\section{控制类指令}

\end{document}